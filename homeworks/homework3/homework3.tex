\documentclass{article}

\def\ParSkip{} 
% Packages
\usepackage{amssymb,amsmath,amsthm,bbm}
\usepackage{verbatim,float,url,dsfont}
\usepackage{graphicx,subfigure,psfrag}
\usepackage{algorithm,algorithmic}
\usepackage{mathtools,enumitem}
\usepackage{multirow}
\usepackage{ragged2e}
\usepackage{xr-hyper}
\usepackage{array}

\usepackage[colorlinks=true,citecolor=blue,urlcolor=blue,linkcolor=blue]{hyperref}
\usepackage[margin=1in]{geometry}
\usepackage[round]{natbib}

\usepackage[utf8]{inputenc} % allow utf-8 input
\usepackage[T1]{fontenc}    % use 8-bit T1 fonts
\usepackage{booktabs}       % professional-quality tables
\usepackage{nicefrac}         % compact symbols for 1/2, etc.
\usepackage{microtype}      % microtypography

\ifdefined\TimesFont 
\usepackage{times} % use times font
\fi

\ifdefined\ParSkip 
\usepackage{parskip} % use par skip
\fi

% Theorems and such
\newtheorem{theorem}{Theorem}
\newtheorem{lemma}{Lemma}
\newtheorem{corollary}{Corollary}
\newtheorem{proposition}{Proposition}
\theoremstyle{definition}
\newtheorem{remark}{Remark}
\newtheorem{definition}{Definition}

% Assumption
\newtheorem*{assumption*}{\assumptionnumber}
\providecommand{\assumptionnumber}{}
\makeatletter
\newenvironment{assumption}[2]{
  \renewcommand{\assumptionnumber}{Assumption #1#2}
  \begin{assumption*}
  \protected@edef\@currentlabel{#1#2}}
{\end{assumption*}}
\makeatother

% Widebar
\makeatletter
\newcommand*\rel@kern[1]{\kern#1\dimexpr\macc@kerna}
\newcommand*\widebar[1]{%
  \begingroup
  \def\mathaccent##1##2{%
    \rel@kern{0.8}%
    \overline{\rel@kern{-0.8}\macc@nucleus\rel@kern{0.2}}%
    \rel@kern{-0.2}%
  }%
  \macc@depth\@ne
  \let\math@bgroup\@empty \let\math@egroup\macc@set@skewchar
  \mathsurround\z@ \frozen@everymath{\mathgroup\macc@group\relax}%
  \macc@set@skewchar\relax
  \let\mathaccentV\macc@nested@a
  \macc@nested@a\relax111{#1}%
  \endgroup
}
\makeatother

% Min and max
\newcommand{\argmin}{\mathop{\mathrm{argmin}}}
\newcommand{\argmax}{\mathop{\mathrm{argmax}}}
\newcommand{\minimize}{\mathop{\mathrm{minimize}}}
\newcommand{\maximize}{\mathop{\mathrm{maximize}}}
\newcommand{\st}{\mathop{\mathrm{subject\,\,to}}}

% Shortcuts
\def\R{\mathbb{R}}
\def\C{\mathbb{C}}
\def\Z{\mathbb{Z}}
\def\N{\mathbb{N}}
\def\E{\mathbb{E}}
\def\P{\mathbb{P}}
\def\T{\mathsf{T}}
\def\Cov{\mathrm{Cov}}
\def\Var{\mathrm{Var}}
\def\indep{\perp\!\!\!\perp}
\def\th{^{\text{th}}}
\def\tr{\mathrm{tr}}
\def\df{\mathrm{df}}
\def\dim{\mathrm{dim}}
\def\col{\mathrm{col}}
\def\row{\mathrm{row}}
\def\nul{\mathrm{null}}
\def\rank{\mathrm{rank}}
\def\nuli{\mathrm{nullity}}
\def\spa{\mathrm{span}}
\def\sign{\mathrm{sign}}
\def\supp{\mathrm{supp}}
\def\diag{\mathrm{diag}}
\def\aff{\mathrm{aff}}
\def\conv{\mathrm{conv}}
\def\dom{\mathrm{dom}}
\def\hy{\hat{y}}
\def\hf{\hat{f}}
\def\hmu{\hat{\mu}}
\def\halpha{\hat{\alpha}}
\def\hbeta{\hat{\beta}}
\def\htheta{\hat{\theta}}
\def\cA{\mathcal{A}}
\def\cB{\mathcal{B}}
\def\cD{\mathcal{D}}
\def\cE{\mathcal{E}}
\def\cF{\mathcal{F}}
\def\cG{\mathcal{G}}
\def\cK{\mathcal{K}}
\def\cH{\mathcal{H}}
\def\cI{\mathcal{I}}
\def\cL{\mathcal{L}}
\def\cM{\mathcal{M}}
\def\cN{\mathcal{N}}
\def\cP{\mathcal{P}}
\def\cS{\mathcal{S}}
\def\cT{\mathcal{T}}
\def\cW{\mathcal{W}}
\def\cX{\mathcal{X}}
\def\cY{\mathcal{Y}}
\def\cZ{\mathcal{Z}}


\title{Homework 3 \\ \smallskip
\large Advanced Topics in Statistical Learning, Spring 2023 \\ \smallskip
Due Friday March 24 at 5pm}
\date{}

\begin{document}
\maketitle
\RaggedRight
\vspace{-50pt}

\section{Carath\'{e}odory's view on sparsity of lasso solutions [10 points]} 

In this exercise, we will prove the fact we cited in lecture about sparsity of
lasso solutions, by invoking Caratheodory's theorem. Let $Y \in \R^n$ be a
response vector, $X \in \R^{n \times d}$ be a predictor matrix, and consider
the lasso estimator defined by solving
\[
\minimize_\beta \; \frac{1}{2} \|Y - X\beta\|_2^2 + \lambda \|\beta\|_1,
\]
for a tuning parameter $\lambda > 0$.

\begin{enumerate}[label=(\alph*)]
\item Let \smash{$\hbeta$} be any solution to the lasso problem. Let
  \smash{$\hat\alpha = \hbeta / \|\hbeta\|_1$}. Prove that \smash{$X
    \hat\alpha$} lies in the convex hull of the vectors 
  \marginpar{\small [2 pts]}
  \[
  \{ \pm X_j \}_{j=1}^d.
  \]
  Note: here $X_j \in \R^n$ denotes the $j\th$ column of $X$.

\item Recall that Carath\'{e}odory's theorem states the following: given any set 
  $C \subseteq \R^k$, every element in its convex hull $\conv(C)$ can be 
  represented as a convex combination of $k+1$ elements of $C$. 

  Use this theorem and part (a) to prove that there exists a lasso solution
  \smash{$\tilde\beta$} with at most $n+1$ nonzero coefficients.  
  \marginpar{\small [2 pts]}

  Hint: start with a generic solution \smash{$\hbeta$}, and use
  Carath\'{e}odory's theorem to construct a coefficient vector
  \smash{$\tilde\beta$} such that (i) the fit is the same, \smash{$X\tilde\beta
    = X\hbeta$}; (ii) the penalty is at worst the same,
  \smash{$\|\tilde\beta\|_1 \leq \|\hbeta\|_1$}; and (iii) \smash{$\tilde\beta$}
  is a nonnegative linear combination of at most $n+1$ of $\pm X_j$,
  $j=1,\dots,d$.  
 
\item Now, assuming $\lambda>0$, use the subgradient optimality condition for 
  the lasso problem to prove that the solution \smash{$\tilde\beta$} from part
  (b) is supported on a subset of 
  \marginpar{\small [3 pts]}
  \[
  \{ \pm X_j \}_{j=1}^d
  \]
  that has affine dimension at most $n-1$.

  Hint: this is similar to the proof of Proposition 1 in the lasso lecture
  notes. Assume that \smash{$\tilde\beta$} is a nonnegative combination of
  exactly $n+1$ of $\pm X_j$, $j=1,\dots,d$. Then one of these $n+1$ vectors, 
  denote it by $s_i X_i$ (where \smash{$s_i = \sign(\tilde\beta_i)$}) can be
  written as a linear combination of the others. Take an inner product with the
  lasso residual and use the subgradient optimality condition for the lasso to
  prove that the coefficients in this linear combination must sum to 1, and
  therefore, $s_i X_i$ is actually an affine combination of the others. Notice
  that this shows the affine span of the $n+1$ vectors in question is
  $(n-1)$-dimensional.     

\item A refinement of Carath\'{e}odory's is as follows: given a set $C \subseteq
  \R^k$, every element in its convex hull $\conv(C)$ can be represented as a
  convex combination of $r+1$ elements of $C$, where $r$ is the affine dimension
  of $\conv(C)$. 

 Use this theorem and part (c) to prove that there exists a lasso solution
  \smash{$\check\beta$} with at most $n$ nonzero coefficients.  
  \marginpar{\small [2 pts]}
\end{enumerate}

\section{Variance of least squares in nonlinear feature models [15 points]} 

\def\asto{\overset{\mathrm{as}}{\to}}
\def\hSigma{\hat\Sigma}

In this exercise, we will examine the variance of least squares (in the
underparametrized regime) and min-norm least squares (in the overparametrized 
regime) in nonlinear feature models. Recall for a response vector $Y \in \R^n$
and feature matrix $X \in \R^{n \times d}$, the min-norm least squares estimator 
\smash{$\hbeta = (X^\T X/n)^+ X^\T Y/n$} has a variance component of its
out-of-sample prediction risk (conditional on $X$) given by:
\begin{equation}
\label{eq:ls_min_var}
\Var_X(\hbeta) = \frac{\sigma^2}{n} \tr (\hSigma^+ \Sigma),
\end{equation}
where \smash{$\hSigma = X^\T X/n$}, and $\sigma^2 = \Var(y_i | x_i)$. This
reduces to \smash{$\Var_X(\hbeta) = \frac{\sigma^2}{n} \tr (\hSigma^{-1}
  \Sigma)$} when \smash{$\hSigma$} is invertible. In lecture, we studied a
linear feature model of the form  
\begin{equation}
\label{eq:linear_features}
X = Z \Sigma^{1/2},
\end{equation}
for a covariance matrix $\Sigma \in \R^{d \times d}$ and a random matrix $Z \in
\R^{n \times d}$ that has i.i.d.\ entries with mean zero and unit variance. When
$\Sigma = I$, which we will assume throughout this homework problem, recall that
we proved that the variance \eqref{eq:ls_min_var} satisfies, under standard
random matrix theory conditions, as $n,d \to \infty$ and $d/n \to \gamma \in 
(0,\infty)$, 
\begin{equation}
\label{eq:ls_min_var_iso_limit}
\Var_X(\hbeta) \asto 
\begin{cases}
\sigma^2 \frac{\gamma}{1-\gamma} & \text{for $\gamma < 1$} \\ 
\sigma^2 \frac{1}{\gamma-1} & \text{for $\gamma < 1$}.
\end{cases}
\end{equation}
(The result for $\gamma < 1$ actually holds regardless of $\Sigma$.) Instead, we
can consider a nonlinear feature model of the form 
\begin{equation}
\label{eq:nonlinear_features}
X = \varphi(Z \Sigma^{1/2} W^\T),
\end{equation}
where $W \in \R^{d \times k}$ is a matrix of i.i.d.\ $N(0,1/k)$ entries, and
$\varphi : \R \to \R$ is a nonlinear function---called the activation function
in a neural network context---that we interpret to act elementwise on $Z
W^\T$. 

There turns to be an uncanny connection between the asymptotic variance in
linear and nonlinear feature models, which will you uncover via simulation in
this homework problem. Attach your code as an appendix to this homework. 

\begin{enumerate}[label=(\alph*)]
\item Fix $n=200$, and let $d=[\gamma n]$ over a wide range of values for
  $\gamma$ (make sure your range covers both $\gamma<1$ and $\gamma>1$). For
  each $n,d$, draw $X$ from the linear feature model \eqref{eq:linear_features}
  with $\Sigma=I$ and your choice of distribution for the entries of
  $Z$. Compute the finite-sample variance \eqref{eq:ls_min_var}, and plot it, as
  a function of $\gamma$, on top of the asymptotic variance curve  
  \eqref{eq:ls_min_var_iso_limit}. To get a general idea of what this should
  look like, refer back to Figure 2 in the overparametrization lecture notes.     
  \marginpar{\small [3 pts]}

\item For the same values of $n,d$, and $k=100$, draw $X$ from the nonlinear
  feature model \eqref{eq:nonlinear_features}, for three different choices of
  $\varphi$:   
  \begin{enumerate}[label=\roman*.]
  \item $\varphi(x) = a_1\tanh(x)$;
  \item $\varphi(x) = a_2(x_+-b_2)$;
  \item $\varphi(x) = a_3(|x|-b_3)$.
  \end{enumerate}
  Here $a_1,a_2,b_2,a_3,b_3$ are constants that you must choose to meet the 
  standardization conditions $\E[\varphi(G)]=0$ and $\E[\varphi(G)^2]=1$, for 
  $G \sim N(0,1)$. Produce a plot just as in part (a), with the finite-sample
  variances for choice of each activation function plotted in a different color,
  on top of the asymptotic variance curve \eqref{eq:ls_min_var_iso_limit} for
  the linear model case. Comment on what you find: do the nonlinear
  finite-sample variances lie close to the asymptotic variance for the linear
  model case?  
  \marginpar{\small [9 pts]}

\item Now use a linear activation function $\phi(x) = ax-b$, and create a plot
  as in part (b) with the same settings (same values of $n,d,k$, and so
  on). What behavior do the finite-sample variances have as a function of 
  $\gamma$? Is this surprising to you? Explain why what you are seeing is
  happening. 
 \marginpar{\small [3 pts]}

\item As a bonus, in light of part (c), elaborate on why the results in part (b)
  are remarkable. 

\item As another (large) bonus, rerun the analysis in this entire problem but
  with a non-isotropic covariance $\Sigma$ in \eqref{eq:linear_features},
  \eqref{eq:nonlinear_features}. Extra bonus points if you properly recompute
  the asymptotic variance curves. 
\end{enumerate}



\bibliographystyle{plainnat}
\bibliography{../../common/ryantibs}

\end{document}